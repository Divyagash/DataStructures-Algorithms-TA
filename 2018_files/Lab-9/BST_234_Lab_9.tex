\documentclass{beamer}
 
\usepackage[utf8]{inputenc}
\usepackage{graphicx}
\usepackage{tabularx}
\usepackage{amssymb}
\usepackage{verbatim}
\usepackage{fancyvrb}
\usepackage{amsmath}
\usepackage{bm}
\usepackage{tikz}
\usetheme{Madrid}
\newcommand{\Perp}{\perp \! \! \! \perp}

\usepackage{listings}
\usepackage{color}

\definecolor{dkgreen}{rgb}{0,0.6,0}
\definecolor{gray}{rgb}{0.5,0.5,0.5}
\definecolor{mauve}{rgb}{0.58,0,0.82}

\lstset{frame=tb,
  language=Java,
  aboveskip=3mm,
  belowskip=3mm,
  showstringspaces=false,
  columns=flexible,
  basicstyle={\small\ttfamily},
  numbers=none,
  numberstyle=\tiny\color{gray},
  keywordstyle=\color{blue},
  commentstyle=\color{dkgreen},
  stringstyle=\color{mauve},
  breaklines=true,
  breakatwhitespace=true,
  tabsize=3
}

\title[BST 234]{BST 234: Lab - 9}
\author[Divy Kangeyan]{Divy Kangeyan}
%\institute{Stat 365R}
\date{\today}

\begin{document}

	%
	
	\begin{frame}
		\titlepage
	\end{frame}
	
	%IF YOU INCLUDE TABLE OF CONTENTS
	%\section[Outline]{}
	%\frame{\tableofcontents}
	
\section{Stability of Algorithm}




\begin{frame}
\frametitle{Minimizing Least Squares}

\begin{itemize}
\item Normal Equations: $A^TAx = A^Tb$
\item If $\kappa(A)$ is large then $\kappa(A^TA)$ will be even larger and near singular matrix
\item Hence inverting $A^TA$ becomes unstable
\end{itemize}

\end{frame}


\begin{frame}
\frametitle{QR decomposition}

\begin{itemize}
\item Given $m \times n$ A, with $m > n$, QR decomposition produce Q, an orthogonal $m \times m$ matrix and R, an $n \times n$ upper triangular matrix such that
   \begin{align}
    A &= Q \begin{bmatrix}
           R \\
           O \\
         \end{bmatrix}
  \end{align}
\item Orthogonal matrix: $Q^T = Q^{-1}$
\item Hence least square problem becomes:
\begin{align}
min ||r||^2_2 &=min  ||b - Ax||^2_2\\
&= min ||b - Q \begin{bmatrix}
           R \\
           O \\
         \end{bmatrix}x ||^2_2\\
&=min ||Q^Tb - \begin{bmatrix}
           R \\
           O \\
         \end{bmatrix}x ||^2_2
\end{align}
\end{itemize}

\end{frame}



\begin{frame}
\frametitle{Getting Q and R}

\begin{itemize}
\item Givens Rotation: Method for introducing zeros in a matrix by rotating the different columns/rows in a matrix
\item Householder transformation: Orthogonal reflection transformation  (Implementation)
\end{itemize}

\end{frame}



\begin{frame}
\frametitle{Householder transformation}

\begin{itemize}
\item Time complexity for Householder transformation: ~$mn^2 - \frac{n^3}{3}$
\item For Given Rotation the time complexity is : ~$3mn^2 - n^3$
\item 50\% more work to do Givens Rotation compared to Householder transformation
\end{itemize}

\end{frame}


\begin{frame}
\frametitle{Solving non-linear equations - Root finding problem}

\begin{itemize}
\item Solution only exist and is unique in very special cases 
\item Bisection Method: Convergence is guaranteed but extremely slow
\end{itemize}

\end{frame}

\end{document}