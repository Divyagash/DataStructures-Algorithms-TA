%%%%%%%%%%%%%%%%%%%%%%%%%%%%%%%%%%%%%%%%%
% Beamer Presentation
% LaTeX Template
% Version 1.0 (10/11/12)
%
% This template has been downloaded from:
% http://www.LaTeXTemplates.com
%
% License:
% CC BY-NC-SA 3.0 (http://creativecommons.org/licenses/by-nc-sa/3.0/)
%
%%%%%%%%%%%%%%%%%%%%%%%%%%%%%%%%%%%%%%%%%

%----------------------------------------------------------------------------------------
%	PACKAGES AND THEMES
%----------------------------------------------------------------------------------------

\documentclass{beamer}

\mode<presentation> {

% The Beamer class comes with a number of default slide themes
% which change the colors and layouts of slides. Below this is a list
% of all the themes, uncomment each in turn to see what they look like.

%\usetheme{default}
%\usetheme{AnnArbor}
%\usetheme{Antibes}
%\usetheme{Bergen}
%\usetheme{Berkeley}
%\usetheme{Berlin}
%\usetheme{Boadilla}
%\usetheme{CambridgeUS}
%\usetheme{Copenhagen}
%\usetheme{Darmstadt}
%\usetheme{Dresden}
%\usetheme{Frankfurt}
%\usetheme{Goettingen}
%\usetheme{Hannover}
%\usetheme{Ilmenau}
%\usetheme{JuanLesPins}
%\usetheme{Luebeck}
\usetheme{Madrid}
%\usetheme{Malmoe}
%\usetheme{Marburg}
%\usetheme{Montpellier}
%\usetheme{PaloAlto}
%\usetheme{Pittsburgh}
%\usetheme{Rochester}
%\usetheme{Singapore}
%\usetheme{Szeged}
%\usetheme{Warsaw}

% As well as themes, the Beamer class has a number of color themes
% for any slide theme. Uncomment each of these in turn to see how it
% changes the colors of your current slide theme.

%\usecolortheme{albatross}
%\usecolortheme{beaver}
%\usecolortheme{beetle}
%\usecolortheme{crane}
%\usecolortheme{dolphin}
%\usecolortheme{dove}
%\usecolortheme{fly}
%\usecolortheme{lily}
%\usecolortheme{orchid}
%\usecolortheme{rose}
%\usecolortheme{seagull}
%\usecolortheme{seahorse}
%\usecolortheme{whale}
%\usecolortheme{wolverine}

%\setbeamertemplate{footline} % To remove the footer line in all slides uncomment this line
%\setbeamertemplate{footline}[page number] % To replace the footer line in all slides with a simple slide count uncomment this line

%\setbeamertemplate{navigation symbols}{} % To remove the navigation symbols from the bottom of all slides uncomment this line
}

\usepackage{graphicx} % Allows including images
\usepackage{booktabs} % Allows the use of \toprule, \midrule and \bottomrule in tables
\usepackage{listings}
%----------------------------------------------------------------------------------------
%	TITLE PAGE
%----------------------------------------------------------------------------------------



\title[BST 234 Lab IV]{BST 234 - Lab IV Complexity Classes and Parallelization} % The short title appears at the bottom of every slide, the full title is only on the title page

\author{Eleni Elia } % Your name
\institute[ ] % Your institution as it will appear on the bottom of every slide, may be shorthand to save space
{
  \\ % Your institution for the title page
\medskip
\textit{ } % Your email address
}
\date{2/14/18}%\date{\today} % Date, can be changed to a custom date
\begin{document}

\begin{frame}
\titlepage % Print the title page as the first slide
\end{frame}

 

%----------------------------------------------------------------------------------------
%	PRESENTATION SLIDES
%----------------------------------------------------------------------------------------

 

\begin{frame}
\frametitle{Complexity Class Definitions}
 \begin{itemize}
 \item (P):Polynomial time algorithms\\
 \begin{itemize}
 \item A decision which may be solved using a deterministic Turing machine in a polynomial amount of time.
 \end{itemize} 
{\small *Deterministic Turing machine: A formal mathematical model for defining an abstract machine capable of simulating any algorithm. Don't worry about knowing this! }
\item (NP):Nondeterministic Polynomial time algorithms
\begin{itemize}
\item Verifying an instance takes polynomial time.
\end{itemize}
{\small *Important to note this does not mean non-polynomial but instead means non-deterministic polynomial!}
\item (NPH):NP-Hard algorithms
Problems that are at least as hard as the hardest problems in NP.\\
{\small Consider an NP-hard algorithm A that can call an oracle to solve A, then it solves B, an NP algorithm, in polynomial time.}
 \end{itemize}
 \end{frame}



\begin{frame}
\frametitle{Complexity Class Definitions continued}
\begin{itemize}

\item (NPC):NP-Complete algorithms
\begin{itemize}
\item Both NP and NP-Hard. Although the verification may be performed in polynomial time,there is no efficient way to locate a solution in the first place.
\end{itemize}

\item (NC):Nick?s Class
\begin{itemize}
\item Algorithms solvable in logarithmic time with a polynomial number of parallel processors. 
\end{itemize}
{\small Intuitively, the logarithmic problem size on a polynomial number processors guarantees an efficient (interesting) speed up.}
\end{itemize}
NC is a subset of P but whether NC=P is unproven, i.e.  there may exist purely sequential algorithms that may not be significantly sped up.
\end{frame}

%------------------------------------------------
\begin{frame}
\frametitle{Complexity Classes Relationship}
 
\begin{figure}
%\includegraphics[width=0.8\linewidth]{compl}
\end{figure}
\end{frame}

 \begin{frame}
 \frametitle{Efficiency and Speedup}
 
 For a sequential algorithm A, the run-time complexity is $T_{A}(n)$.\\
 Parallelized on p processors, the run-time complexity is  $T_{p}(n)$.\\
 \vspace{1cm}
 
\textbf{ Algorithm efficiency: }$$E_{p}(n)=\frac{T_{1}(n)}{pxT_{p}(n)} $$
\textbf{ Algorithm speed-up:}$$S_{p}(n)=\frac{T_{A}(n)}{T_{p}(n)}$$
 \end{frame}
%------------------------------------------------


\begin{frame}
 \frametitle{Threading versus Multiprocessing}
Various parallel architectures were introduced in the lecture notes. \\
Notably, we introduced the parallel random access memory(PRAM) architecture.  While useful in theory, real hardware is different and we achieve parallelization through two main concepts: 
\begin{itemize}
\item Threading
\item Multiprocessing
\end{itemize}
\end{frame}

\begin{frame}
 \frametitle{Threading versus Multiprocessing}
 
\begin{columns}[c] % The "c" option specifies centered vertical alignment while the "t" option is used for top vertical alignment

\column{.45\textwidth} % Left column and width

 
\begin{block}{Threading:}
A single shared memory is accessible to individual treads/sequences\\
{\color{blue} -Multiple people adding pieces to the same puzzle}
\end{block}

\begin{block}{Multiprocessing:}
Each process is given its own memory space\\
{\color{blue}-Each person working on their own part of the puzzle}
\end{block}
 
\column{.5\textwidth} % Right column and width

\begin{figure}
%\includegraphics[width=0.56\linewidth]{puzzle1}
\end{figure}

\begin{figure}
%\includegraphics[width=0.65\linewidth]{puzzle}
\end{figure}

 

\end{columns}
 

\end{frame}


\begin{frame}
 \frametitle{Techniques for Parallelization}
 
 \begin{itemize}
 \item Embarrassingly Parallel
 \item Divide-and-Conquer Strategies
 \item Pipelined Computations
 \end{itemize}
\end{frame}

\begin{frame}
\frametitle{Parallelization in Python}
\begin{block}{A few options for parallelizing code }
\begin{itemize}
\item xmlrpclib
\item threading
\item multiprocessing
\item pp (parallel python)
\end{itemize} 
\end{block} 

\begin{itemize}
\item Easy enough to augment existing code by dropping in the parallelization
\item Important to note that each python interpreter has a single global interpreter lock (GIL)
\end{itemize}  
\end{frame}




\begin{frame}
\frametitle{Global Interpreter Lock - a thread-safe mechanism}
\begin{itemize}
 

\item Executes only one statement at a time (so-called serial processing, or single-threading) to prevent conflicts between threads. 
\item We can spawn multiple subprocesses to avoid some of the GIL?s disadvantages.
\end{itemize}
\begin{figure}
%\includegraphics[width=0.8\linewidth]{gil_source_pythoncertgithub}
\end{figure}
\url{https://www.youtube.com/watch?v=ph374fJqFPE}
\url{http://www.dabeaz.com/GIL/gilvis/fourthread.html}
\end{frame}
 


\begin{frame}
\frametitle{Vectorizing}
Similar to R, see the numpy module for vectorize
\end{frame}

\begin{frame}
\frametitle{References}
\url{https://pymotw.com/3/multiprocessing/index.html} %module-multiprocessing
\url{https://pymotw.com/3/threading/index.html} %module-threading
\url{http://www.davekuhlman.org/python_multiprocessing_01.html}
\url{http://www.dabeaz.com/GIL/gilvis/index.html}
\end{frame}


 
%http://sebastianraschka.com/Articles/2014_multiprocessing.html
%https://docs.python.org/dev/library/multiprocessing.html
%https://www.ploggingdev.com/2017/01/multiprocessing-and-multithreading-in-python-3/
 
%----------------------------------------------------------------------------------------

\end{document}