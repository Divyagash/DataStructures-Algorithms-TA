\documentclass{beamer}
 
\usepackage[utf8]{inputenc}
\usepackage{graphicx}
\usepackage{tabularx}
\usepackage{amssymb}
\usepackage{verbatim}
\usepackage{fancyvrb}
\usepackage{amsmath}
\usepackage{bm}
\usepackage{tikz}
\usetheme{Madrid}
\newcommand{\Perp}{\perp \! \! \! \perp}


\title[BST 234]{BST 234: Lab - 7}
\author[Divy Kangeyan]{Divy Kangeyan}
%\institute{Stat 365R}
\date{\today}

\begin{document}
	%
	
	\begin{frame}
		\titlepage
	\end{frame}
	
	%IF YOU INCLUDE TABLE OF CONTENTS
	%\section[Outline]{}
	%\frame{\tableofcontents}
	
\section{Pseudo Random Generation (PRNG)}




\begin{frame}
\frametitle{Random numbers}

\begin{itemize}
\item Efficiently generating independent random variables from the uniform distribution is important in permutation tests and many other useful instances.
\item Random variable from uniform distributions can be transformed into other continuous random variables, i.e. Inverse Probability-Integral Transform
\item Useful in simulations, transaction, cryptography etc. 
\end{itemize}

\end{frame}






\begin{frame}
\frametitle{Properties of a good Pseudo Random Number Generators (PRNG)}

\begin{itemize}
\item Uniformity
\item Independence
\item Passes all the diehard tests:
\begin{itemize}
\item Birthday spacings test
\item Overlapping mutations test
\item Ranks of matrices test
\item Monkey test
\item Count the 1's test
\item Parking lot test
\item \textbf{Minimum distance test}
\item Random Sphere test
\item Squeeze test
\item \textbf{Overlapping Sums test}
\item Runs test
\item The craps test
\end{itemize}

\end{itemize}

\end{frame}



\begin{frame}
\frametitle{Properties of a good Pseudo Random Number Generators (PRNG) - Cont'd}

\begin{itemize}
\item Replication (reason for generating pseudo random numbers instead of random numbers)
\item Cycle length
\item Speed
\item Memory usage
\item Parallel implementation
\item cryptographically secure

\end{itemize}

\end{frame}



\begin{frame}
\frametitle{PRNG: Mid-square method}

Algorithm:
\begin{itemize}
\item Start with a 4-digit seed ($z_0$)
\item Square it to get 8 digit number, pad with zeros if necessary
\item Take middle 4 digits from the 8-digit number generated
\item Divide the 4-digit number by 10000 to generate Uniform RV

\end{itemize}
\textit{Python Demonstration}

\end{frame}


\begin{frame}
\frametitle{Linear Congruential Generator}
Produces a sequence of numbers between $0$ and $m-1$

Algorithm:
\begin{itemize}
\item Start with the seed $z_0$
\item $z_n = (az_{n-1}+c)$ mod $m$ $n = 1, 2, ...$
\item To get Uniform RV $u_n = z_n/m$
\item Choice of a, c and m are important

\end{itemize}
\textit{Python Demonstration}
\end{frame}

\begin{frame}
\frametitle{Other Congruential Generator}

\begin{itemize}
\item Multiplicative Congruential Generators ($z_n = az_{n-1}$)
\begin{itemize}
\item Doesn't have full period
\end{itemize}
\item Additive Congruential Generators ($z_n = z_{n-1} + z_{n-k}$)
\begin{itemize}
\item Can have very long period upto $m^k$
\end{itemize}

\end{itemize}

\end{frame}



\begin{frame}
\frametitle{Mersenne Twister}

\begin{itemize}
\item Current gold standard to generate PRN
\item Invented by two Japanese scientists Makoto Matsumoto and Takuji Nishimura
\item Passes all diehard tests 
\item Has very long period of $2^{19937} - 1$

\end{itemize}
\textit{Python Demonstration for two diehard tests for PRN generated by Mersenne Twister}
\end{frame}





\begin{frame}
\frametitle{Truncation Error}

\begin{itemize}
\item Introduced by algorithm via problem simplification, e.g. series truncation, iterative process truncation etc.
\item For example several functions can be approximated by Taylor series expansion
\item \textit{Python demonstration}
\end{itemize}

\end{frame}


\begin{frame}
\frametitle{Relative Error}

\begin{itemize}
\item Absolute error in basic arithmetic operation: $(\tilde{x} * \tilde{y}) - (x * y)$  $*:{+, -, x, /}$
\item Relative error in basic arithmetic operation: $\frac{(\tilde{x} * \tilde{y}) - (x * y)}{(x * y)}$  $*:{+, -, x, /}$
\item \textit{Python demonstration}
\end{itemize}

\end{frame}

\end{document}