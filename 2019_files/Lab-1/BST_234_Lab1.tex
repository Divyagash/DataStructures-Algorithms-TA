\documentclass{beamer}
 
\usepackage[utf8]{inputenc}
\usepackage{graphicx}
\usepackage{tabularx}
\usepackage{amssymb}
\usepackage{verbatim}
\usepackage{fancyvrb}
\usepackage{amsmath}
\usepackage{bm}
\usepackage{tikz}
\usetheme{Madrid}
\newcommand{\Perp}{\perp \! \! \! \perp}

\title[BST 234]{BST 234: Lab - 1}
\author[Divy Kangeyan]{Divy Kangeyan}
%\institute{Stat 365R}
\date{\today}

\begin{document}
	%
	
	\begin{frame}
		\titlepage
	\end{frame}
	
	%IF YOU INCLUDE TABLE OF CONTENTS
	%\section[Outline]{}
	%\frame{\tableofcontents}
	
\section{Python}

\begin{frame}
\frametitle{Python usage}

\begin{itemize}
\item For command prompt: Python in the terminal or script
\item For interactive use: IPython notebook (Jupyter etc.)
\end{itemize}

Quick tips on Python

\begin{itemize}
\item Zero-based indexing
\item \texttt{math}, \texttt{numpy}, \texttt{scipy} are useful modules for scientific computing and \texttt{matplotlib} is useful for visualization
\item Several data structures available such as: lists, dictionaries, tuples, arrays etc. 
\end{itemize}


\begin{center}
\small

\normalsize
\end{center}

\begin{center}
%\includegraphics[width=9cm, height=4cm]{DR_ex}
\end{center}


\end{frame}


\begin{frame}
\frametitle{Normalized Floating Point Representation}

\begin{center}
$x = \pm m * b^{\pm e}$
\end{center}

\begin{itemize}
\item base $b \in \mathbb{N}$ and $b > 1$
\item mantissa $m = m_1b^{-1}+...+m_rb^{-r} \in \mathbb{R}$
\item exponent $e = e_{s-1}b^{s-1}+...+e_0b^0 \in \mathbb{N}$
\item digits $m_i, e_i \in 0,...,b-1$
\item significant digits $s \in \mathbb{N}$ and $r \in \mathbb{N}$
\end{itemize}

\end{frame}

\begin{frame}
\frametitle{Practice: Normalized Floating Point Representation}

Express the number x = 10 in normalized floating point format for the base b = 2: 
\pause

\begin{itemize}
\item Find k such that $b^k \leq x \leq b^{k+1}$
$$2^3 \leq 10 \leq 2^4$$ \pause
\item Factor x into $b^k, b^{k-1},...$
$$10 = 2^3 \times 1 + 2^2 \times 0 + 2^1 \times 1 + 2^0 \times 0$$ \pause
\item Add terms and factor out $b^{k+1}$
$$10 = 2^4(2^{-1} \times 1 + 2^{-2} \times 0 + 2^{-3} \times 1 + 2^{-4} \times 0)$$ \pause
\item Answer: $x = (.101)_2 * 2^4$

\end{itemize}

\end{frame}

\begin{frame}
\frametitle{Practice: Normalized Floating Point Representation}

Express the number x = 100 in normalized floating point format for the base b = 3: 
\pause

\begin{itemize}
\item Find k such that $b^k \leq x \leq b^{k+1}$
\item Factor x into $b^k, b^{k-1},...$
\item Add terms and factor out $b^{k+1}$ \pause
\item Answer: $x = (.10201)_3 * 3^5$

\end{itemize}

\end{frame}


\begin{frame}
\frametitle{Machine Precision}
Definition:
$$eps := \frac{1}{2}b^{-r+1}$$

For the IEEE-format:
$$eps_{IEEE} \leq \frac{1}{2}2^{-51}$$

\begin{itemize}
\item \textit{Python demonstration}
\end{itemize}
\pause


$$\frac{7}{3} = 1.0010101010101010101010101010101010101010101010101011*2^1$$
 $$\frac{4}{3} = 1.0101010101010101010101010101010101010101010101010101*2^0$$

$$\frac{7}{3} -\frac{4}{3} =  1.0000000000000000000000000000000000000000000000000001*2^0$$
$$\frac{7}{3} -\frac{4}{3} - 1 = 2^{-52} = \epsilon$$



\end{frame}

\begin{frame}
\frametitle{Floating point arithmetic}
\begin{itemize}
\item Since floating point arithmetic is inherently approximate and not exact following symbols are used: $\oplus,\ominus, \otimes,\oslash$
\item  $(x \oplus y) \oplus z \neq x \oplus (y \oplus z) $ (Associative law doesn't hold)
\item $(x \oplus y) \otimes z \neq (x \otimes z) \oplus (y \otimes z) $ (Distributive law doesn't hold)
\item $x \oplus y = x$ for $|y| \leq \frac{|x|}{b}eps$
\item \textit{Python demonstration}
\end{itemize}
\end{frame}


\begin{frame}
\frametitle{Practice: Prove the identity}

Prove that $u \ominus v = -(v \ominus u)$ based on the following identities:
$$u \oplus v = v \oplus u$$\\
$$u \oplus 0 = u $$


\pause

\begin{align}
u \ominus v &= u \oplus -v\\
&= -v \oplus u \\
&= - (v \oplus -u)\\
&= -(v \ominus u)
\end{align}

\end{frame}


\end{document}